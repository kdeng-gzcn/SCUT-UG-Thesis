% \chapter{结论}
% \label{cha:experiment}
% \section*{论文工作总结}
% ……
% \section*{工作展望}
% ……

\chapter{结论}

本文研究了 GraphEM 算法在线性高斯状态空间模型(LGSSM)参数估计中的应用。GraphEM 是一种新颖的算法,它将先验知识整合进期望最大化(EM)框架中。通过引入图推理与正则化技术,我们对原始 GraphEM 方法进行了扩展,使其能够处理更广泛的图结构与正则化策略,包括 Laplace、Gaussian 及混合 Laplace+Gaussian 先验。

一个具有前景的结果是,GraphEM 框架能够无缝集成多种正则项。Laplace 与 Gaussian 先验的组合实现了两者优势的“无折中”叠加,而不引入额外的优化开销。这一特性为未来的研究开辟了新方向:通过累积多样化的正则项,可以在不增加计算复杂度的前提下,迭代地逼近最优解。

% \section*{未来工作}

未来的研究方向包括发展自适应正则化策略,根据观测数据与图结构动态调整 Laplace 与 Gaussian 先验的权重。此外,进一步研究 GraphEM 在其他图类型上的表现,如随机图、分层图与动态图,有助于验证其通用性。对混合 Laplace+Gaussian 正则方法的收敛性与鲁棒性进行理论分析,也将有助于深入理解其有效性。最后,将 GraphEM 应用于金融、气候建模与社交网络分析等实际问题中,有望展示其在参数估计中的实用价值与潜在影响。
