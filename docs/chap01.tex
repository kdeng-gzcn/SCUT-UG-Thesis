%%
% 引言或背景
% 引言是论文正文的开端,应包括毕业论文选题的背景、目的和意义;对国内外研究现状和相关领域中已有的研究成果的简要评述;介绍本项研究工作研究设想、研究方法或实验设计、理论依据或实验基础;涉及范围和预期结果等。要求言简意赅,注意不要与摘要雷同或成为摘要的注解。
%%

\chapter{绪论}
%定义,过去的研究和现在的研究,意义,与图像分割的不同,going deeper
\label{cha:introduction}
% \section{引言}
% \label{sec:background}
% 当今社会,科技的飞速发展为大家提供了快捷与舒适,但与此同时也增添了在信息安全上的危险。在过去的二十几年来,我们通过数字密码来鉴别身份,但是随着科技的发展,不法分子借用高科技犯罪的案例年年增高,密码被盗的情况时常发生。因此,怎样科学准确的辨别每一个人的身份则成为当今社会的重要问题。
% \section{研究背景}
% \label{sec:related_work}
% 随着科技的日益发展,传统的密码因为记忆的繁琐以及容易被盗,似乎已经不再能满足这个通信发达的社会的需求。人们急需一种更便捷而且辨识度更高的方式来辨识身份。循着便捷与辨识度高这两个约束条件\overcite{ref1},我们联想到的便是存在于每个人身上的生物特征,所以基于每个人身上不同的生物特征而研究的鉴别技术现在成为了身份辨别技术上的主流。

% \section{研究现状}
% 笔迹获取的方式有两种,所以鉴别方式也分为离线鉴别和在线鉴别\overcite{ref2,ref3}。在线鉴别是采用专用的数字板来实时收集书写信号。由文献\cite{ref4,ref5,ref6,ref7}可知,因为信号是实时采集的,所以能采集的数据不仅包括笔迹序列,而且可以采集到书写时的加速度、压力、速度等丰富有用的动态信息。

% \section{论文结构}
% 本文分为四章。其中第一章简述了笔迹识别的研究背景和意义以及笔迹识别的基础知识等。第二章节从卷积神经网络的发展历史、网络结构、学习规律三方面详细的讲述了卷积网络的基础知识。第三章针对本文中的手写数字及写字人实验具体设计卷积神经网络的网络结构以及训练过程。第五章节是手写数字识别及写字人识别实验的结果与分析。

\section{问题背景}
状态空间模型 (SSM) 广泛应用于工程~\cite{hamilton1994state, kim2017state}、环境科学~\cite{choi2023urban_phd_environ}、信号处理~\cite{dabush2024kalman_tracking_network_dynamic} 等各个领域的随机过程建模。SSM 利用隐藏变量来表示底层的马尔可夫过程,使其成为分析时间序列数据的强大工具。

当模型参数已知时,贝叶斯最优滤波器和平滑器为求解状态空间模型~\cite{Särkkä_Svensson_2023_textbook} 提供了理论框架。这些工具能够进行准确的预测,为工业应用中的决策和战略制定提供参考。具体而言,贝叶斯滤波器能够以高可靠性预测状态空间模型中的长期行为。在线性高斯状态空间模型 (LGSSM) 中,该过程允许一个解析解,详见~\cite{Särkkä_Svensson_2023_textbook}。对于非线性状态空间模型,粒子滤波器提供了一种实用的解决方案。

本文重点介绍线性高斯状态空间模型 (LGSSM)。在 LGSSM 中,诸如转移矩阵之类的参数对于理解卡尔曼滤波器的行为至关重要。转移矩阵封装了不同的马尔可夫链,其准确估计至关重要。然而,在实际场景中,模型参数通常未知,但具有有用的属性~\cite{watts1998collective},因此参数估计是一项至关重要的任务。

\subsection{现有方法}
LGSSM 中的参数估计主要有三种方法。第一种方法是直接优化~\cite{gupta1974computational},它使用基于梯度的方法迭代逼近最优模型参数。第二种方法是期望最大化 (EM) 方法~\cite{neal1998view},它优化目标函数的上限以控制其行为~\cite{shumway1982approach, Särkkä_Svensson_2023_textbook}。第三种方法是马尔可夫链蒙特卡洛 (MCMC),它通过采样来近似模型参数的分布。然而,这些方法都不是直接针对模型参数的最大后验 (MAP) 估计。结合先验分布对于理解系统隐藏的动态通常至关重要。

一种名为 GraphEM 的新算法通过将先验知识集成为正则化来解决这一限制~\cite{chouzenoux2020GraphEM}。 GraphEM 以 EM 方法为基础,利用其快速收敛和理论保证,同时结合了对模型参数(尤其是转移矩阵)的图推理。通过结合不同的马尔可夫过程,GraphEM 在估计现实世界中的转移矩阵(通常具有稀疏性等特性)方面表现出卓越的性能。

\subsection{创新方法}
GraphEM 引入了转移矩阵的先验知识,通过稀疏性正则化增强了算法。受转移矩阵图推理视角的启发,我们探索了除稀疏性之外的其他图属性~\cite{granger1969investigating, ravazzi2017learning}。具体而言,我们通过引入新的先验知识类型来扩展 GraphEM,包括高斯先验以及结合拉普拉斯分布和高斯分布的混合先验~\cite{zou2005regularization}。通过对各种图类型(例如小世界图、无标度图、循环图和星形图)进行大量实验,我们证明了所提变体的多功能性和鲁棒性。

本文的贡献如下:
\begin{itemize}
    \item \textbf{深入探索基于图结构的统计推断}:我们对过渡矩阵背景下的图推理进行了全面的分析,重点突出了不同图结构(例如小世界图、无标度图)对参数估计的影响。该分析为理解 GraphEM 在不同拓扑条件下的行为提供了新的见解。
    \item \textbf{提出GraphEM的新变体}:我们提出并实现了两种新的 GraphEM 变体:一种融合了高斯先验,另一种结合了拉普拉斯先验和高斯先验。这些变体扩展了原始方法的能力,使其能够处理除稀疏性之外的更广泛的图属性。
    \item \textbf{提出新的正则化策略}:通过比较拉普拉斯、高斯和混合先验的性能,我们确定了每种正则化策略的优势场景。此外,我们的分析表明,通过在我们的优化框架内组合不同类型的正则化项,有可能逼近最优解,而不会产生额外的计算成本。
\end{itemize}

\section{相关文献}
卡尔曼滤波器及其扩展的理论基础已非常完善,~\cite{Särkkä_Svensson_2023_textbook}提供了全面的推导和分析。

卡尔曼滤波器的参数估计已得到广泛研究,其中期望最大化 (EM) 算法是最广泛使用的方法之一。EM 算法(如~\cite{Särkkä_Svensson_2023_textbook} 所述)通过迭代优化似然函数,为估计模型参数提供了一个稳健的框架。然而,传统的 EM 方法通常缺乏融入先验知识的能力,而这对于提高实际应用中的估计精度至关重要。

参数估计领域的最新进展主要集中在通过集成图推理技术来增强 EM 算法。 ~\cite{chouzenoux2020GraphEM} 的开创性工作提出了 GraphEM,这是一种将转移矩阵的先验知识作为正则化项的新方法。该方法利用转移矩阵的稀疏结构,从而提升了各种应用中的估计性能。GraphEM 的有效性已在多个领域得到证实,包括股票市场预测 ~\cite{sharma2021stock_prediction}、社交媒体趋势分析 ~\cite{dahiya2021twitter_prediction} 和气候建模 ~\cite{elvira2023GraphEM_in_climate}。

GraphEM 的进一步扩展已在时间序列预测和非凸优化领域得到探索。例如,~\cite{elvira2022GraphEM_TSP} 将 GraphEM 应用于旅行商问题 (TSP) 场景,而 ~\cite{chouzenoux2023graph_it} 则研究了其在非凸设置下的性能。此外,~\cite{chouzenoux2024non_markov_lagrangeEM} 提出了一种基于拉格朗日的 GraphEM 变体,用于非马尔可夫系统,进一步拓宽了其适用性。

近期的研究也致力于提升 GraphEM 的计算效率和可扩展性。例如,~\cite{cox2022sparj_alg_journal} 引入了 SPARJ 算法,提高了大规模系统中图推理的效率。该算法在 ~\cite{cox2023sparj_alg} 和 ~\cite{cox2024graphgrad_TSP} 中得到了进一步扩展,其中集成了基于梯度的方法来处理高维转移矩阵。此外,~\cite{chouzenoux2024DGLASSO} 提出了一种联合估计转移矩阵和噪声协方差的新方法,显著提高了估计精度。

与此同时,其他研究人员也在探索状态空间模型中参数估计的替代方法。例如,~\cite{tsampourakis2022alr_changepoint} 提出了一种用于变点检测的自适应调优方法,而~\cite{sharma2023stock_later} 将 GraphEM 与循环神经网络 (RNN) 相结合,用于股票市场预测。此外,~\cite{cui2024inference_non_linear} 使用哈密顿蒙特卡罗方法研究了非线性系统的推理方法,~\cite{mohanty2025iclr_granger} 则在图模型的背景下探索了格兰杰因果关系。